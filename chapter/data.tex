\chapter{Data}

The study by Artuc, Rijkers and Porto used several data sets to enable the heterogenous view of households in regard of their
expenditures, auto-consumption and incomes.
One of the main datasets used are household surveys of 54 developing countries which were taken between the years 1998 and
2015. The dataset covers all low income countries, where appropriate household survey data was available, and additionally
contains the majority of lower middle income countries.
%grafik Übersicht Survey Data
\begin{figure}[ht!]
    \centering
    \includegraphics*[scale=0.5]{graphics/householdsurvey1.jpg}
    \includegraphics*[scale=0.5]{graphics/householdsurvey2.jpg}
    \caption{Household Survey Data by Artuc, Rijkers and Porto 2019}
    \label{fig:household survey data}
\end{figure}\\
The household survey data was harmonized by adopting and improving templates based on Nicita, Olarreaga and Porto from 2014.
There are three templates used covering the expenditure of households, their auto-consumption and their income. Those templates 
contain several categories, which hold a unique two or four-digit ID. \\
The first template used is the expenditure template, which
contains the categories agriculture/food, manufacturing/household items, services and other expenditures. The group agriculture/food
is split into staple and non-staple food. The second group, manufacturing/household items contains energy, textiles/apparel,
electric/electronics, household items/furniture and other physical goods. The service group contains transportation, health,
education, communication and other services. Other expenditures shows remittances/transfers given, investment of any sort,
festivities and other disbursement.\\
%grafik expenditure template
\begin{figure}[h!]
    \centering
    \includegraphics*[width=\textwidth]{graphics/templ1.jpg}
    \caption{Expenditure template based on Deaton modified by Artuc, Rijkers and Porto 2019}
    \label{fig:template exp}
\end{figure}\\
The next template regarding the auto-consumption of each household is split into agriculture/food and other goods. Agriculture/food
contains staple and non-staple food and other goods contains energy, gathering, other goods collected for free and other goods
produced and consumed within the household.\\

The final template showcasing the income of each householkd is split into agriculture/food, wages, sales of goods/services and 
transfers. The group agriculture/food contains the same data as mentioned above. Wages contain agriculture/forestry/fishing,
mining/oil/gas extraction, manufacturing, construction, transportation/communication/electric/gas/sanitary, wholesale/retail,
finance/insurance/real estate, entertainment services, professional services and public administration. The group sales of
goods/services contains the same categories as the wages group. The last group transfers includes remittances/transfers received,
profits of investment, government transfers, non-governmental transfers and others. \\
%grafik auto consumption template

Besides the survey data, the authors of the paper also used datasets regarding the quantity and value of traded goods as well 
as import tariffs on goods for each country. The data regarding the quantity and value of traded goods for each country is 
sourced from the COMTRADE-Database by the Trade Statistics Section of United Nations Statistics Division.
The tariff data is sourced from the TRAINS-Database by Trade Information Section of United Nations Conference on Trade and
Development. Both datasets use the HS 6-digit code to uniquely identify every possible traded good. This HS 6-digit code
was used to merge with the unique two and four-digit codes from the adopted survey templates. This harmonization leads to the 
advantage over the most other studies regarding the impact of trade liberalization as household heterogeneity is granted in 
the data for the model to use.
