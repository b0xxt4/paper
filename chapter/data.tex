\section{Data used by Artuc et al.}
The study by Artuc, Rijkers and Porto used several data sets to enable the heterogenous view of households in regard of their
expenditures, auto-consumption and incomes. One of the main datasets used were household surveys of 54 developing countries, 
which were taken between the years 1998 and 2015. The dataset covered all low income countries, where appropriate household survey
data was available, and additionally contained the majority of lower middle income countries as well in Table \ref{tab:household survey data}
\parencite[pp.~10-11]{Artuc.2019}.
%grafik Übersicht Survey Data
\begin{table}[ht!]
    \centering
    \includegraphics*[scale=0.7]{graphics/householdsurvey1.jpg}
    \includegraphics*[scale=0.7]{graphics/householdsurvey2.jpg}
    \caption{Household Survey Data \parencite[pp.~56-57]{Artuc.2019}}
    \label{tab:household survey data}
\end{table}\\
To enable a comparability of the household survey data, the household survey data was harmonized by adopting and improving templates based 
on the work of \cite{Nicita.2014}. Using those templates, they were able to map the information at the highest 
level of disaggregation to the different homogeneous categories \autocite[p.~255]{Nicita.2014}.
While Nicita, Olarreaga and Porto only developed and used two templates regarding the incomes and expenditures of the households, Artuc, Rijkers
and Porto added another template to harmonize information regarding the auto-consumption of the households as well, which explained 23 percent
of the household income \autocites[p.~255]{Nicita.2014}[p.~11]{Artuc.2019}.\\
There were three templates used, which were covering the expenditure of households, their auto-consumption and their income. Those templates 
contained several categories, which held a unique two or four-digit ID as seen in Table \ref{tab:template exp} for example \Autocite[p.~10]{Artuc.2019}.\\
The first template was the expenditure template, which contained the categories agriculture/food, manufacturing/household items, services 
and other expenditures. The group agriculture/food is split into staple and non-staple food and accounts on average for 45 percent of all 
household spending across all countries \autocite[p.~12]{Artuc.2019}. The second group, manufacturing/household items contained energy, 
textiles/apparel, electric/electronics, household items/furniture and other physical goods. The service group contained transportation, health,
education, communication and other services. Other expenditures shows remittances/transfers given, investment of any sort, festivities and 
other disbursement \parencite[p.~58]{Artuc.2019}.\\
%grafik expenditure template
\begin{table}[h!]
    \centering
    \includegraphics*[width=\textwidth]{graphics/templ1.jpg}
    \caption{Modified Expenditure template based on \cite{Nicita.2014}\parencite[p.~58]{Artuc.2019}}
    \label{tab:template exp}
\end{table}\\
The next template regarding the auto-consumption of each household was split into agriculture/food and other goods. Agriculture/food
contained staple and non-staple food and other goods contains energy, gathering, other goods collected for free and other goods
produced and consumed within the household \parencite[p.~58]{Artuc.2019}.\\
The final template, showcasing the income of each household, was split into agriculture/food, wages, sales of goods/services and 
transfers. The group agriculture/food included the same data as mentioned above. Wages contained agriculture/forestry/fishing,
mining/oil/gas extraction, manufacturing, construction, transportation/communication/electric/gas/sanitary, wholesale/retail,
finance/insurance/real estate, entertainment services, professional services and public administration. The group sales of
goods/services contained the same categories as the wages group. The last group transfers included remittances/transfers received,
profits of investment, government transfers, non-governmental transfers and others \parencite[p.~59]{Artuc.2019}. \\
%grafik auto consumption template

Besides the survey data, the authors of the paper also used datasets regarding the quantity and value of traded goods as well 
as import tariffs on goods for each country. The data regarding the quantity and value of traded goods for each country was 
sourced from the COMTRADE-Database by the Trade Statistics Section of United Nations Statistics Division.
The tariff data was sourced from the TRAINS-Database by Trade Information Section of United Nations Conference on Trade and
Development. Both datasets used the HS 6-digit code to uniquely identify every possible traded good. This HS 6-digit code
was used to merge with the unique two and four-digit codes from the adopted survey templates. This harmonization led to the 
advantage over the most of other studies regarding the impact of trade liberalization as household heterogeneity was granted in 
the data for the model to use \parencite[p.~10]{Artuc.2019}. 
