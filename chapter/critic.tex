\section{Critical aspects}

To determine the significance of the study by \cite{Artuc.2019}, a closer review regarding possible weaknesses of the
study has to be done.\\

To start the critical review of the study, first, the data must be inspected. One of the main issues or possible weaknesses
of the data is the range of years, in which the household surveys were taken. As the earliest household surveys were taken in 
1998 for Burundi, Rwanda, The Gambia, the most recent one was taken in 2015 in Liberia. The outdated part of the data might dilute the results and
assumptions regarding the impact of liberalization for those "outdated" countries as the time between the usage of those
single household surveys and the time they are used in the model ranges up to twenty-one years \parencite[pp.~56-57]{Artuc.2019}.\\ 
In those decades, worldwide economic shifts, economic crises and other factors appeared and could change the household surveys' 
outcome a lot. It is also unknown if the authors of this study adapted the survey data by including the inflation of the single countries. 
While the data definitely showed possible directions for trade-offs between income gains and equality gains, the authors could
have tried to add commonly known economic developments into the data as the inflation rate for example, to increase the 
relevance of their results.\\
Another possible weakness in the usage of survey data is the accuracy of the household's answer and therefore the significance 
of the data itself. Several authors and politicians mentioned the lack of some features and the relatively high error rates and 
the even higher non-answer rates for conducted household surveys. Therefore, the literature and policy try to overcome those
disadvantages by linking survey data with administrative data. However, especially in developing countries, which were the
main focus of this study, administrative data is lacking \parencite[pp.~255-256]{Meyer.2015}.\\
Another potential source of the needed data could result from a new concept of machine learning predictions, where information about poverty,
wealth and possibly income and consumption can be sourced from mobile phone data. By using collected metadata from mobile phones for an algorithm 
to predict outcomes like poverty and wealth, up-to-date data can be collected within 3-4 weeks and costs of only \$ 12,000. This method could improve
the current lack of relevance of household survey data for the developing world, but is not implemented yet by most countries \parencite[p.~1076]{Blumenstock.2015}.\\ 
Therefore, the authors could not have used a better possible dataset for
especially the heterogeneous analysis of households' incomes and expenditures. This can be also confirmed by the authors themselves
as they updated this study in 2022 with the same dataset of household surveys \parencite[pp.~14-15]{Artuc.2022}. The lack in the recurrence of 
household surveys in developing countries while still being used as main source, leads to the conclusion that household survey data is still 
the most detailed data possible, especially in the case of developing countries where other possible data like administrative data is still an 
exception.\\

As the household survey data seems to fit for the study, the remaining parts of the dataset need to be inspected.
First to be reviewed is the used data sourced from the COMTRADE-Database, which covers the quantity and the value of the traded goods. While it
 is seen as a viable choice for sourcing unit values and quantities of traded goods, the aggregation in the COMTRADE-Database can include a 
 bias. \cite[p.~107]{Berthou.2011} claim, that the 'procedure may bias unit value when some quantities are missing at a higher level of disaggregation
(e.g. missing quantity increases unit value after aggregation).'\\
Another weakness of this data lies in the used estimation process to counter
the lack of quantity information. About 60 percent of the estimated values are based on the Standard Unit Value by product category, which leads
to a mismatch of the estimated unit value and the countries' characteristics like production costs or specialization of production \parencite[p.~107]{Berthou.2011}.
Therefore, \Cite{Artuc.2019} should use another source for the quantity and value of traded goods like the Traded Unit Values database, which
offers a higher reliability of the unit values to improve the accuracy for the results of their work in total \parencite[p.~113]{Berthou.2011}.\\

The remaining third part of the data is sourced from the TRAINS-Database and contains information regarding tariffs and the countries' trade
policies. While this data is used a lot in literature regarding the impacts of certain trade policy measures, there seems to be no critical
aspect given in recent literature, as the last negative remark is given back in 2001, that the data is not complete and shows incomplete time series
\parencite[p.~6]{Olarreaga.2001}. This could lead to the assumption, that the TRAINS data seems to be reliable and complete as of today
and therefore was the right data source to be used by \cite{Artuc.2019}.\\ 

As the authors themselves mentioned, their model relies on the extended agricultural household model by \cite{Singh.1986} and 
\cite{Benjamin.1993}. This leads to an inability of the model to capture the differential impacts on the impacts of capital vs labor or skilled vs unskilled
labor. However, not only the model can not capture those details as the data itself lacks information about capital and corporate
profits, so that just by choosing this dataset, the authors decided against the differential analysis of these points
as they argue, that especially when regarding the poverty or the welfare from countries, household survey data is still the most frequently
used source of data \parencite[p.~9]{Artuc.2019}.\\

The final point, which could be misleading, is the assumption about the results of this study, that trade liberalization is an
easy solution for enhancing the social welfare of countries. This study tried to predict the impact of trade liberalization
based on the assumption, that those welfare enhancing effects especially for the low income household are instantly receivable.
However, there is the thought in literature, that the impact of trade liberalization on low income households depends on the
environment and policies of the country itself. Therefore, a need for additional policies can be needed to enhance or even
enable the positive impact for the poor households of a country. This important role of complementary politics is also underlined
by the evidence, that shows that especially low income households have lower chances per se to protect themselves socially and
to enhance the possibility to exploit the anticipated benefits caused by the liberalization \parencite[p.~107]{Winters.2004}. 
Even under the circumstances above, the author of this literature concludes, that while trade liberalization might not be 
the most powerful or direct tool to address welfare of a country, it is one of the easiest tools to use with a positive outcome.
So in the end, even while the isolated view of trade liberalization can lead to not reaching the anticipated benefits, it is
still a useful and easy to execute tool to increase the welfare of low income households and battle poverty of a country \parencite[p.~108]{Winters.2004}.\\