\chapter{Critical aspects}

To determine the significance of the study by Artuc, Rijkers and Porto, a closer review regarding possible weaknesses of the
study has to be done. 
To start the critical review of the study, first, the data must be inspected. One of the main issues or possible weaknesses
of the data is the range of years, in which the household surveys were taken. As the earliest household survey was taken in 
1998 for Rwanda, the most recent one was taken in 2015 in Liberia. The outdated part of the data might dilute the results and
assumptions regarding the impact of liberalization for those "outdated" countries as the time between the usage of those
single household surveys and the time they are used in the model ranges up to twenty-one years. In those decades, worldwide
economic shifts, economic crises and other factors could appear and could change the household surveys' outcome a lot. It is
also unknown if the authors of this study adapted the survey data by including the inflation of the single countries. 
While the data definitely shows possible directions for trade-offs between income gains and equality gains, the authors could
have tried to add commonly known economic developments into the data as the inflation rate for example, to increase the 
significance of their results.
% quelle????

Another possible weakness in the usage of survey data is the accuracy of the household's answer and therefore the significance 
of the data itself. Several authors and politicians mentioned the lack of some features and relatively high error rates and 
even higher non-answer rates for conducted household surveys. Therefore, the literature and policy tries to overcome those
disadvantages by linking survey data with administrative data. However, especially in developing countries, which were the
main focus of this study, administrative data is lacking. Therefore, the authors could not use a better possible dataset for
especially the heterogeneous analysis of households' incomes and expenditures. This can be also confirmed by the authors themselves
as they updated this study in 2022 with the same dataset of household surveys. The lack in the recurrence of household surveys
in developing countries leads to the conclusion, that household survey data is still the most detailed data possible, especially
in the case of developing countries as other possible data like administrative data is still an exception.
% reference journal of economic perspectives

As the authors themselves mentioned, their model relies on the extended agricultural household model by Singh etc. This leads
to an inability of the model to capture the differential impacts on the impacts of capital vs labor or skilled vs unskilled
labor. However, not only the model can not capture those details as the data itself lacks information about capital and corporate
profits, so that just by choosing this dataset, the authors decided against the differential analysis of these points.
% quelle?

Another thing which could be misleading is the assumption about the results of this study, that trade liberalization is an
easy solution for enhancing the social welfare of countries. This study tried to predict the impact of trade liberalization
based on the assumption, that those welfare enhancing effects especially for the low income household are instantly receivable.
However, there is the thought in literature, that the impact of trade liberalization on low income households depends on the
environment and policies of the country itself. Therefore, a need for additional policies can be needed to enhance or even
enable the positive impact for the poor households of a country. This important role of complementary politics is also underlined
by the evidence, that shows that especially low income households have lower chances per se to protect themselves socially and
to enhance the possibility to exploit the anticipated benefits by the liberalization. 
Even under the circumstances above, the author of this literature concludes, that while trade liberalization might not be 
the most powerful or direct tool to address welfare of a country, it is one of the easiest tools to use with a positive outcome.
So in the end, even while the isolated view of trade liberalization can lead to not reaching the anticipated benefits, it is
still a useful and easy to execute tool to increase the welfare of low income households and battle poverty of a country.\\
% winters quelle

