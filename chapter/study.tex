\section{Methodology of Artuc et al.}

To use the data explained above, Artuc, Rijkers and Porto needed a model to study the welfare effects of tariff changes. In the
first instance, they adopted an extended agricultural household model to define the household welfare based on the work of \cite{Singh.1986} 
and \cite{Benjamin.1993}. The authors then derived the welfare effects using first order
approximations based on the work of \cites{Deaton.1989}{Porto.2006}{Nicita.2014}.

To determine the impacts of price changes and furthermore on the welfare effects for households, the authors first defined a 
maximized nominal income function for the household 
\begin{equation} \label{eq:equation1}
    y^h(\mathbf{p}, v^h)= w^h + \sum_{ i}\pi_{i}^{h}(\mathbf{p})-T^h+\Omega^h,
\end{equation}
where the household income \(y^h\) depends on the vector of prices \(p_{i}\) and fixed amount of resources \(v^h\).
The labor income of household \(h\) as \(w^h\) is only from the labor market and \(\pi_{i}^{h}\) are farm enterprise profits
obtained from selling the good \(i\).  Governmental taxes paid are represented as \(T^h\). Other transfers and other income are
showcased in \(\Omega^h\) \parencite[pp.~4-5]{Artuc.2019}. \\
To also take the expenditures into consideration, the household expenditure function was defined as 
\begin{equation} \label{eq:equation2}
    e(\mathbf{p}, u^h)= \sum_{i}p_{i}c_{i}^h(\mathbf{p}).
\end{equation}
In this equation, \(p_{i}\) is the price for good \(i\) and \(u^h\) is the required utility for the optimal consumption
\(c_{i}^h\) \parencite[p.~5]{Artuc.2019}.\\
While the income and expenditures of the household were already defined, the expenditures caused by trade could now be summarized
in one equation. Therefore, the authors referenced to \cite{Dixit.1980} and \cite{Anderson.1996} to define their
trade expenditure function as 
\begin{equation} \label{eq:equation3}
    V^h(\mathbf{p}, v^h, u^h)= y^h(\mathbf{p}, v^h) - e(\mathbf{p}, u^h).
\end{equation} 
The authors also referred to \cite{Porto.2006} while explaining, that the traditional expenditure function is defined as \(e^h-y^h\), but
by swapping the terms, they could see the results as changes in real household income \parencite[p.~5]{Artuc.2019}.\\

To obtain estimates of welfare effects, which were applicable with the above explained data, the authors proposed two aspects
to reach those goals. The first proposition assumed that the household is the price taker in consumer, producer and labor markets.
Therefore, the impact of a price change on the household welfare could be defined as
\begin{equation} \label{eq:equation4}
    \frac{dV_{i}^h}{e^h}=((\phi_{i}^h-s_{i}^h)+\phi_{w}^h \frac{\partial w^h}{\partial p_{i}} \frac{p_{i}}{w^h}) d\ln p_{i} -
    \frac{dT^h}{e^h}.
\end{equation}
The monetary transfer needed by household \(h\) to enable the same utility \(u^h\) as before the price change is showcased as 
\(dV_{i}^h\). The share of the traded good \(i\) is \(s_{i}^h\), while the share from the sales of good \(i\) is \(\phi_{i}^h\).
The labor income share is defined as \(\phi_{w}^h\) \parencite[p.~6]{Artuc.2019}.\\

The second proposition contained multiple assumptions. First it assumed that the goods are homogenous and that the 
targeted countries are rather small and are therefore externally facing the international prices of \(p_{i}^*\). There was also
the assumption of the perfect price transmission from tariffs to domestic prices. Finally, they assumed, that the loss of public
revenue caused by the tariff cuts is compensated with the help of income tax increases. Based on this proposition, the estimable
welfare effects were given as
\begin{equation} \label{eq:equation5}
    \frac{dV_{i}^h}{e^h}=((\phi_{i}^h-s_{i}^h)+\phi_{wi}^h) \frac{\tau_{i}}{1+\tau_{i}}+\Psi_{i}^h.
\end{equation}
The share of labor income \(\phi_{wi}^h\) is now specified for the sector \(i\) and \(\Psi_{i}^h\) is the tax increase for the
household \(h\). The level of tariff protection in sector \(i\) is assumed to be \(\tau_{i}\).
This equation only worked under the assumption, that the country reduced its own tariffs individually, therefore assuming a full
unilateral tariff liberalization. While the possibility for a full import tariff liberalization could be showcased in the equation,
the data did not contain information regarding the pass-through elasticities and therefore needed to be simplified as shown
above \parencite[p.~7]{Artuc.2019}.\\
Finally, to measure the welfare effects of the entire tariff protection and not only for single sectors, the equation
could be summed up as 
\begin{equation} \label{eq:equation7}
    \hat{V}^h = \frac{dV^h}{e^h} = \sum_{i} \frac{dV_{i}^h}{e^h}.
\end{equation}
The proportional change of real household income can be displayed as \(\hat{V}^h\). This equation can also be used to estimate
the counterfactual real income under the assumption that \(x_{0}^h\) is the observed ex-ante level of real household income to
define the equation as 
\begin{equation} \label{eq:equation8}
    \hat{x}_{1}^h = x_{0}^h(1+\hat{V}^h).
\end{equation}
In this equation, \(\hat{x}_{1}^h\) is the counterfactual real income.
As the authors used an agricultural household model, there were some differences to standard trade models since the data in form
of household surveys did not contain returns to capital or corporate profits. The authors named the Stolper-Samuelsen effects
as an example of effects, which show the differential impacts on returns to capital vs labor or to skilled vs unskilled labor, 
which could not be captured in this study. However, they argued that topics like poverty, inequality and household welfare are
usually based on household surveys, therefore it was beneficial to use a model, which is able to use this dataset. Another
benefit was named in the household heterogeneity regarding the income and the consumption, which led to results regarding the
total gains as well as for inequality costs since the model could differentiate between rich and poor households \parencite[pp.~8-9]{Artuc.2019}.\\

As seen in the data, there was still a need of weighted average tariff rates for every single category of the harmonized 
dataset. Those tariff rates were defined as 
\begin{equation} \label{eq:equation9}
    \tau_{i}= \sum_{c,n \in i} \tau_{c,n} \frac{m_{c,n}}{\sum_{c,n \in i}m_{c,n}}.
\end{equation}
Every category of the HS 6-digit classification is represented as \(n\) for the 2- and 4-digit category \(i\) from the survey. 
The imports of good \(n\) are for the country \(c\) are given as \(m_{c,n}\). The resulting average tariffs were 14.4 percent
for non-staple agricultural goods and 10.8 percent for staple agricultural goods. The category manufactures yielded an
average tariff of 10.9 percent.
To determine the impact of the elimination of those tariffs on the prices, the authors set the equation as
\begin{equation} \label{eq:equation10}
    \Delta \ln p_{i} = \frac{p_{i}^*-p_{i}^*(1+\tau_{i})}{p_{i}^*(1+\tau_{i})} = -\frac{\tau_{i}}{1+\tau_{i}}
\end{equation}
and referred back to the assumption of the full price transmission \parencite[p.~11]{Artuc.2019}.\\

The authors reviewed the equation \ref{eq:equation4} with weighted household survey data. Excluding the top and bottom 
0.5 percentile to reduce the measurement error, they showed averages for six biggest household expenditures, which were
Staple Agriculture, Non-Staple Agriculture, Manufactured Goods, Non-Traded Goods, Other Goods and Home Consumption. The biggest
expenditure was seen in the category food with an average of 45 percent of all household spendings. The authors argued that this
was expected as the survey data held an average poverty rate of 35 percent and an average GDP per capita of US\$ 1879 \parencite[p.~12]{Artuc.2019}.\\

Regarding the compensation of the tariff revenue loss, the authors assumed before that the government would impose a proportional
income tax, which was displayed as 
\begin{equation} \label{eq:equation11}
    \psi_{i}^h = -\frac{\tau_ {i}}{1+\tau_{i}} \frac{M_{i}}{\sum_{h}y^h}.    
\end{equation}
The revenue loss is shown as \(\psi_{i}^h\) and the value of imports is shown as \(M_{i} = p_{i}^*(1+\tau_{i})m_{i}\) \parencite[p.~13]{Artuc.2019}.\\

To define the income gains from trade, which are portrayed as the proportional change in aggregate household real expenditures
after the import tariff liberalization as in \cite{Arkolakis.2012}, the equation was 
\begin{equation} \label{eq:equation12}
    G = \frac{\sum_{h}(x_{1}^h-x_{0}^h)}{\sum_{h}x_{0}^h} = \frac{x_{0}^h}{\sum_{h}x_{0}^h}\hat{V}^h.
\end{equation}
As \(\hat{V}^h\) was explained above as the proportional change in real expenditures of household \(h\), \(G\) can bee seen as 
the weighted average of the welfare effects. Using this equation on the harmonized survey data, tariff liberalization induced 
a gain of 2.5 percentage points in real expenditures for 45 countries, where the gains were positive. Ten countries faced a loss of
an average of 0.9 percent of real expenditures. In total, the average for all countries was 1.9 percent, which implicated that
the developing countries seem to gain from trade \parencite[p.~14]{Artuc.2019}.\\
%insert table 4 and 5?

To inspect the distributional effects of trade, the authors first estimated kernel averages of the gains from trade dependent on
the initial well-being of the household per capita expenditure. Then they estimated bivariate kernel densities of the distribution
of gains from trade and household per capita expenditure. Using the pro-poor Index of \cite{Nicita.2014},
they divided the countries in to two groups, pro-poor and pro-rich. The pro-poor Index is the proportional change of low
20 percent income household minus the high 20 percent income households. If the index is positive, the tariff liberalization
can be seen as pro-poor as the poor household gain proportionally more than the rich households. The opposite would indicate
a countries' tariff liberalization as pro-rich. As a result, 17 countries were classified as pro-poor and the remaining 37 countries
had a pro-rich gain distribution \parencite[p.~15]{Artuc.2019}.
%\begin{figure}[h!]
%    \centering
 %   \includegraphics*[scale=0.5]{graphics/propoor.jpg}
 %   \caption{Example countries for Pro-Poor Bias by Artuc, Rijkers and Porto from 2019}
  %  \label{fig:propoor}
%\end{figure}

%\begin{figure}[h!]
   % \centering
    %\includegraphics*[scale=0.5]{graphics/prorich.jpg}
    %\caption{Example countries for Pro-Rich Bias by Artuc, Riijkers and Porto from 2019}
    %\label{fig:prorich}
%\end{figure}
To tackle the possible trade-off between income inequality and average incomes, the authors referred to the Atkinson social 
welfare function from \cite{Atkinson.1970}:
\begin{equation} \label{eq:equation13}
    W =\frac{1}{H}\sum_{h}\frac{(x^h)^{1-\varepsilon}}{1-\varepsilon}. 
\end{equation}
The social welfare is \(W\) and \(\varepsilon \neq 1\) is the inequality aversion parameter, which can be seen as a weighting for
the well-being of low income households. One benefit of this equation was that it could be also defined as 
\begin{equation} \label{eq:equation14}
    W = \mu * (1-I),
\end{equation}
in which \(\mu\) is mean income and \(I\) is the atkinson inequality index, which is defined as 
\begin{equation} \label{eq:equation15}
    I = 1-(\frac{1}{H}\sum_{h=1}^H (x^h/\mu)^{1-\varepsilon})^{1/(1-\varepsilon)}.
\end{equation}
The authors used the two equations above to define an equation for the inequality adjusted income gains as 
\begin{equation} \label{eq:equation16}
    G(\varepsilon)= \frac{W_{1}(\varepsilon)-W_{0}(\varepsilon)}{W_{0}(\varepsilon)}.
\end{equation}
The ex-ante social welfare is \(W_{0}\) and \(W_{1}(\varepsilon)\) is the counterfactual social welfare under the tariff liberalization.
As above-mentioned in equation 14, this equation can be decomposed into 
\begin{equation} \label{eq:equation17}
    G(\varepsilon)= G(0) + \frac{\mu_{1}}{\mu_{0}} \frac{I_{0}(\varepsilon)-I_{1}(\varepsilon)}{1-I_{0}(\varepsilon)},
\end{equation}
to display that the inequality adjusted gains are simply the income gains of trade \(G(0)\) with an adaptation for changes in inequality 
or also called equality gains in the second term of the equation \parencite[pp.~17-19]{Artuc.2019}. \\
To determine the possibility of a trade-off between the income gains \(G(0)\) and the equality gains, which are represented as 
\(\frac{\mu_{1}}{\mu_{0}} \frac{I_{0}(\varepsilon)-I_{1}(\varepsilon)}{1-I_{0}(\varepsilon)}\), those terms have to have
opposite signs. This would be the case if the trade would decrease the equality gains, which would mean the inequality would
increase, while the income gains would increase. The other possible case for a trade-off would be an increase in equality gains
and a decrease in income gains. Those trade-offs appeared for the majority of the countries, in case of this study for 45 countries.
The remaining 9 countries did not face a trade-off. However, 27 of the trade-off countries could face, depending on the inequality
aversion parameter \(\varepsilon\), severe reversals in the preference of trade policy \parencite[p.~19]{Artuc.2019}. \\

As mentioned above, the authors differentiated the results into three categories. The first category were no trade-off countries, which means
that those countries faced the same development in the income and the equality gains in case of a liberalization. In total, the authors saw
8 countries in this category, from which 4 countries showed a positive development of income and equality gains in favor of liberalization. 
Those 4 countries were the Central African Republic, Guinea-Bissau, Jordan and Yemen. However, the left 4 countries were Comoros, Ghana, 
Madagascar and Rwanda and would face losses that were independent of the inequality aversion parameter \(\varepsilon\) and therefore indicated 
a favor for protectionism \parencite[p.~20]{Artuc.2019}. \\

The next category was seen as trade-off countries without trade policy preference reversals. As mentioned before, only 18 countries faced a 
trade-off soft enough to indicate for only one trade policy for all inequality aversion parameter \(\varepsilon\). From those countries, 
only 2 countries showcased a favor for protectionism while the remaining countries tended to a strong domination in regard of a trade liberalization
 \parencite[pp.~20-21]{Artuc.2019}. \\
%insert table/figure?

The last group of countries were countries, which faced severe trade policy preference reversals caused by the trade-off. Those
countries preferences changed in regard of the weighting of the well-being of low income households, which was defined as \(\varepsilon\).
Those 27 countries showcased that for a low \(\varepsilon\), most countries favored the protectionism and with an ongoing increase
of \(\varepsilon\) the countries tended to favor the liberalization instead \parencite[pp.~22-24]{Artuc.2019}. \\
%insert table/figure?
The authors viewed the results as follows, even with an \(\varepsilon = 1.5\), which is close to the Gini coefficient,
there were 30 countries with a trade-off in favor of liberalization. 28 of the studies' countries would experience inequality
adjusted gains from liberalization and only seven countries with trade-offs would prefer the protectionism because in their case, 
tariffs would lead to higher inequality-adjusted welfare. However, in total, the trade liberalization was expected to increase 
the welfare in 39 countries and reduce it in only 9 countries for plausible levels of the inequality aversion parameter 
\(\varepsilon\), which could be seen in Figure \ref{fig:tradereso} as well. Therefore, the authors asked themselves why the majority of countries still prioritize protectionism over 
liberalization, they can not answer this question directly and see a possible answer in the political, economic considerations
of the single countries. In total those results showed, that protectionism seems to be loaded with high costs in regard of the 
social welfare \parencite[pp.~24-26]{Artuc.2019}. \\
\begin{figure}[h!]
    \centering
    \includegraphics*[width=\textwidth]{graphics/trade-off-reso.jpg}
    \caption{Scatter plot to display the countries favorable trade policy \parencite[p.~43]{Artuc.2019}}
    \label{fig:tradereso}
\end{figure}\\

To underline their results, the authors used two different permutations of their model to check the model's robustness, which they
confirmed by the very similar pattern of the inequality-adjusted gains. Another point of their robustness check was in the form of 
reevaluation of two alternative assumptions about the tariff redistribution. The first assumption of changes was the exclusion of 
tariff revenue from the model, however the correlation of equality gains and income gains was still negative. The second
reevaluation was a change in the compensation of the tariff losses by governmental progressive taxes instead of proportional
taxes, however the pattern did not show any impact and the trade-off countries were still dominated in favor of 
liberalization \parencite[pp.~28-29]{Artuc.2019}. \\
Another form of robustness test was done by using the model to evaluate three different protectionist scenarios instead,
the first one increased the tariffs by 10 percent uniformly, the second one was a relative increase of 10 percent and the last
one was an increase to 62.4 percent for all tariffs. However, all three scenarios confirmed the results, that in this case
43 to 48 countries' inequality-adjusted income gains would decrease with increased protection as shown in Table \ref{tab:protectionist}. 
This underlined the assumption, that the majority of countries is actually financially hurt by their ongoing protectionism \parencite[pp.~29-30]{Artuc.2019}.
\begin{table}[h!]
    \centering
    \includegraphics*[scale=0.4]{graphics/protectionist.jpg}
    \caption{Results of three Protectionist Scenarios \parencite[p.~55]{Artuc.2019}}
    \label{tab:protectionist}
\end{table}\\
Finally, to summarize their study's results, the authors came to the conclusion, that their model confirmed the negative
correlation between income gains and equality gains, which leads to a trade-off in the case of liberalization. While liberalization
led to income gains in 45 countries in their study, 9 countries would faced losses instead. To finish, they gave their thought that
based on their study, the majority of developing countries would profit of liberalization and that the ongoing trend of protectionism
is causing significant welfare losses in developing countries \parencite[p.~30]{Artuc.2019}. 