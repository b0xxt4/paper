\chapter{Study}

To use the data explained above, Artuc, Rijkers and Porto need a model to study the welfare effects of tariff changes. In the
first instance, they adopt an extended agricultural household model to define the household welfare based on the work of Singh, 
Squire and Strauss from 1986 and Benjamin from 1993. The authors then derive the welfare effects using first order
approximations based on the work of Deaton from 1989, Porto, 2006; Nicita, Olarreaga and Porto, 2014.

To determine the impacts of price changes and furthermore on the welfare effects for households, the authors first defined a 
maximized nominal income function for the household 
\begin{equation} \label{eq:equation1}
    y^h(\mathbf{p}, v^h)= w^h + \sum_{ i}\pi_{i}^{h}(\mathbf{p})-T^h+\Omega^h,
\end{equation}
where the household income \(y^h\) depends on the vector of prices \(p_{i}\) and fixed amount of resources \(v^h\).
The labor income of household \(h\) as \(w^h\) is only from the labor market and \(\pi_{i}^{h}\) are farm enterprise profits
obtained from selling good \(i\).  Governmental taxes paid are represented as \(T^h\). Other transfers and other income are
showcased in \(\Omega^h\). \\
To also take the expenditures into consideration, the household expenditure function is defined as 
\begin{equation} \label{eq:equation2}
    e(\mathbf{p}, u^h)= \sum_{i}p_{i}c_{i}^h(\mathbf{p}).
\end{equation}
In this equation, \(p_{i}\) is the price for good \(i\) and \(u^h\) is the required utility for the optimal consumption
\(c_{i}^h\).\\
While the income and expenditures of the household are already defined, the expenditures caused by trade can now be summarized
in one equation. Therefore, the authors reference to Dixit and Norman (1980) and Anderson and Neary (1996) and use their
trade expenditure function as 
\begin{equation} \label{eq:equation3}
    V^h(\mathbf{p}, v^h, u^h)= y^h(\mathbf{p}, v^h) - e(\mathbf{p}, u^h).
\end{equation} 
The authors refer to Porto (2006) while explaining, that the traditional expenditure function is defined as \(e^h-y^h\), but
by swapping the terms, they can see the results as changes in real household income.\\

To obtain estimates of welfare effects which are applicable with the above explained data, the authors propose two aspects
to reach those goals. The first proposition assumes, that the household is the price taker in consumer, producer and labor markets.
Therefore, the impact of a price change on the household welfare could be defined as
\begin{equation} \label{eq:equation4}
    \frac{dV_{i}^h}{e^h}=((\phi_{i}^h-s_{i}^h)+\phi_{w}^h \frac{\partial w^h}{\partial p_{i}} \frac{p_{i}}{w^h}) d\ln p_{i} -
    \frac{dT^h}{e^h}.
\end{equation}
The monetary transfer needed by household \(h\) to enable the same utility \(u^h\) as before the price change is showcased as 
\(dV_{i}^h\). The share of the traded good \(i\) is \(s_{i}^h\), while share from the sales of good \(i\) is \(\phi_{i}^h\).
The labor income share is defined as \(\phi_{w}^h\).\\

The second proposition contains multiple assumptions. First it assumes that the goods are homogenous and that the 
targeted countries are rather small and are therefore facing exogenously the international prices of \(p_{i}^*\). There is also
the assumption of the perfect price transmission from tariffs to domestic prices. Finally, they assume, that the loss of public
revenue caused by the tariff cuts is compensated with the help of income tax increases. Based on this proposition, the estimable
welfare effects are given as
\begin{equation} \label{eq:equation5}
    \frac{dV_{i}^h}{e^h}=((\phi_{i}^h-s_{i}^h)+\phi_{wi}^h) \frac{\tau_{i}}{1+\tau_{i}}+\Psi_{i}^h.
\end{equation}
The share of labor income \(\phi_{wi}^h\) is now specified for the sector \(i\) and \(\Psi_{i}^h\) is the tax increase for the
household \(h\). The level of tariff protection in sector \(i\) is assumed to be \(\tau_{i}\).
This equation only works under the assumption, that the country reduced its own tariffs individually, therefore assuming a full
unilateral tariff liberalization. While the possibility for a full import tariff liberalization could be showcased in the equation,
the data does not contain information regarding the pass-through elasticities and therefore needs to be simplified as shown
above.
Finally, to measure the welfare effects of the entire tariff protection and not only for single sectors, the equation
can be summed up as 
\begin{equation} \label{eq:equation7}
    \hat{V}^h = \frac{dV^h}{e^h} = \sum_{i} \frac{dV_{i}^h}{e^h}.
\end{equation}
The proportional change of real household income can be displayed as \(\hat{V}^h\). This equation can also be used to estimate
the counterfactual real income under the assumption that \(x_{0}^h\) is the observed ex-ante level of real household income to
define the equation as 
\begin{equation} \label{eq:equation8}
    \hat{x}_{1}^h = x_{0}^h(1+\hat{V}^h).
\end{equation}
In this equation, \(\hat{x}_{1}^h\) is the counterfactual real income.
As the authors use an agricultural household model, there are some differences to standard trade models since the data in form
of household surveys does not contain returns to capital or corporate profits. The authors name the Stolper-Samuelsen effects
as an example of effects, which show the differential impacts on returns to capital vs labor or to skilled vs unskilled labor, 
which can not be captured in this study. However, they argue that topics like poverty, inequality and household welfare are
usually based on household surveys, therefore it is beneficial to use a model, which is able to use this dataset. Another
benefit is named in the household heterogeneity regarding the income and the consumption, which leads to results regarding the
total gains as well as for inequality costs since the model can differentiate between rich and poor households.\\

As seen in the data, there is still a need of weighted average tariff rates for every single category of the harmonized 
dataset. Those tariff rates can be defined as 
\begin{equation} \label{eq:equation9}
    \tau_{i}= \sum_{c,n \in i} \tau_{c,n} \frac{m_{c,n}}{\sum_{c,n \in i}m_{c,n}}.
\end{equation}
Every category of the HS 6-digit classification is represented as \(n\) for the 2- and 4-digit category \(i\) from the survey. 
The imports of good \(n\) are for the country \(c\) are given as \(m_{c,n}\). The resulting average tariffs are 14.4 percent
for non-staple agricultural goods and 10.8 percent for staple agricultural goods. The category manufactures yields an
average tariff of 10.9 percent.
To determine the impact of the elimination of those tariffs on the prices, the authors refer back to the assumption of the 
full price transmission in equation 6
to set the equation as
\begin{equation} \label{eq:equation10}
    \Delta ln p_{i} = \frac{p_{i}^*-p_{i}^*(1+\tau_{i})}{p_{i}^*(1+\tau_{i})} = -\frac{\tau_{i}}{1+\tau_{i}}.
\end{equation}\\

The authors review the equation \ref{eq:equation4} with weighted household survey data. Excluding the top and bottom 
0.5 percentile to reduce the measurement error, they show averages for six biggest household expenditures, which are
Staple Agriculture, Non-Staple Agriculture, Manufactured Goods, Non-Traded Goods, Other Goods and Home Consumption. The biggest
expenditure is seen in the category food with an average of 45 percent of all household spendings. The authors argue that this
was expected as the survey data holds an average poverty rate of 35 percent and an average GDP per capita of US\$ 1879.\\

Regarding the compensation of the tariff revenue loss, the authors assumed before that the government would impose a proportional
income tax, which is displayed as 
\begin{equation} \label{eq:equation11}
    \psi_{i}^h = -\frac{\tau_ {i}}{1+\tau_{i}} \frac{M_{i}}{\sum_{h}y^h}.    
\end{equation}
The revenue loss is shown as \(\psi_{i}^h\) and the value of imports is shown as \(M_{i} = p_{i}^*(1+\tau_{i})m_{i}\).

To define the income gains from trade, which are portrayed as the proportional change in aggregate household real expenditures
after the import tariff liberalization as in Arkolakis, Costinot and Rodriguez-Clare from 2012, the equation is 
\begin{equation} \label{eq:equation12}
    G = \frac{\sum_{h}(x_{1}^h-x_{0}^h)}{\sum_{h}x_{0}^h} = \frac{x_{0}^h}{\sum_{h}x_{0}^h}\hat{V}^h.
\end{equation}
As \(\hat{V}^h\) was explained above as the proportional change in real expenditures of household \(h\), \(G\) can bee seen as 
the weighted average of the welfare effects. Using this equation on the harmonized survey data, tariff liberalization induces 
a gain of 2.5 percentage points in real expenditures for 45 countries, where the gains are positive. Ten countries face a loss of
an average of 0.9 percent of real expenditures. In total, the average for all countries is 1.9 percent, which implicates that
the developing countries seem to gain from trade.
%insert table 4 and 5?

To inspect the distributional effects of trade, the authors first estimate kernel averages of the gains from trade dependent on
the initial well-being of the household per capita expenditure. Then they estimate bivariate kernel densities of the distribution
of gains from trade and household per capita expenditure. Using the pro-poor Index of Nicita, Olarreaga and Porto from 2014,
they divide the countries in to two groups, pro-poor and pro-rich. The pro-poor Index is the proportional change of low
20 percent income household minus the high 20 percent income households. If the index is positive, the tariff liberalization
can be seen as pro-poor as the poor household gain proportionally more than the rich households. The opposite would indicate
a countries' tariff liberalization as pro-rich. As a result, 17 countries are classified as pro-poor and the remaining 37 countries
would have a pro-rich gain distribution.
\begin{figure}[h!]
    \centering
    \includegraphics*[scale=0.5]{graphics/propoor.jpg}
    \caption{Example countries for Pro-Poor Bias by Artuc, Rijkers and Porto from 2019}
    \label{fig:propoor}
\end{figure}\\

\begin{figure}[h!]
    \centering
    \includegraphics*[scale=0.5]{graphics/prorich.jpg}
    \caption{Example countries for Pro-Rich Bias by Artuc, Riijkers and Porto from 2019}
    \label{fig:prorich}
\end{figure}\\
To tackle the possible trade-off between income inequality and average incomes, the authors refer to the Atkinson social 
welfare function from 1970:
\begin{equation} \label{eq:equation13}
    W =\frac{1}{H}\sum_{h}\frac{(x^h)^{1-\varepsilon}}{1-\varepsilon}. 
\end{equation}
The social welfare is \(W\) and \(\varepsilon \neq 1\) is the inequality aversion parameter, which can be seen as a weighting for
the well-being of low income households. One benefit of this equation is that it can be also defined as 
\begin{equation} \label{eq:equation14}
    W = \mu * (1-I),
\end{equation}
in which \(\mu\) is mean income and \(I\) is the atkinson inequality index, which is defined as 
\begin{equation} \label{eq:equation15}
    I = 1-(\frac{1}{H}\sum_{h=1}^H (x^h/\mu)^{1-\varepsilon})^{1/(1-\varepsilon)}.
\end{equation}
The authors use the two equations above to define a equation for the inequality adjusted income gains as 
\begin{equation} \label{eq:equation16}
    G(\varepsilon)= \frac{W_{1}(\varepsilon)-W_{0}(\varepsilon)}{W_{0}(\varepsilon)}.
\end{equation}
The ex-ante social welfare is \(W_{0}\) and \(W_{1}(\varepsilon)\) is the counterfactual social welfare under the tariff liberalization.
As above-mentioned in equation 14, this equation can be decomposed into 
\begin{equation} \label{eq:equation17}
    G(\varepsilon)= G(0) + \frac{\mu_{1}}{\mu_{0}} \frac{I_{0}(\varepsilon)-I_{1}(\varepsilon)}{1-I_{0}(\varepsilon)}.
\end{equation}
To determine the possibility of a trade-off between the income gains \(G(0)\) and the equality gains which are represented as 
\(\frac{\mu_{1}}{\mu_{0}} \frac{I_{0}(\varepsilon)-I_{1}(\varepsilon)}{1-I_{0}(\varepsilon)}\), those terms have to have
opposite signs. This would be the case if the trade would decrease the equality gains, which would mean the inequality would
increase, while the income gains would increase. The other possible case for a trade-off would be an increase in equality gains
and a decrease in income gains. Those trade-offs appear for the majority of the countries, in case of this study for 45 countries.
The remaining 9 countries do not face a trade-off. However, 27 of the trade-off countries can face, depending on the inequality
aversion parameter \(\varepsilon\), severe reversals in the preference of trade policy. \\

As mentioned above, the authors differentiate the results into three categories. The first category is seen as trade-off
countries without trade policy preference reversals. As mentioned before, only 18 countries face a trade-off soft enough to 
indicate for only one trade policy for all inequality aversion parameter \(\varepsilon\). From those countries, only 2 countries
showcase a favor for protectionism while the remaining countries tend a strong domination in regard of a trade liberalization.
%insert table/figure?

The second group of countries are countries which face severe trade policy preference reversals caused by the trade-off. Those
countries preferences change in regard of the weighting of the well-being of low income households, which is defined as \(\varepsilon\).
Those 27 countries showcase that for a low \(\varepsilon\), most countries favor the protectionism and with an ongoing increase
of \(\varepsilon\) the countries tend to favor the liberalization. \\
%insert table/figure?
The authors view the results as follows, even with an \(\varepsilon = 1.5\), which is close to the Gini coefficient,
there are 30 countries with a trade-off in favor of liberalization. 28 of the studies' countries would experience inequality
adjusted gains from liberalization and only 7 countries with trade-offs would prefer the protectionism because in their case, 
tariffs would lead to higher inequality-adjusted welfare. However, in total, the trade liberalization is expected to increase 
the welfare in 39 countries and reduce it in only 9 countries for plausible levels of the inequality aversion parameter 
\(\varepsilon\), which can be seen in Figure \ref{fig:tradereso} as well. Therefore, the authors ask themselves why the majority of countries still prioritize protectionism over 
liberalization, they can not answer this question directly and see a possible answer in the political, economic considerations
of the single countries. In total those results show, that protectionism seems to be loaded with high costs in regard of the 
social welfare. \\
\begin{figure}[h!]
    \centering
    \includegraphics*[width=\textwidth]{graphics/trade-off-reso.jpg}
    \caption{Scatter plot to display the countries favorable trade policy by Artuc, Rijkers and Porto from 2019}
    \label{fig:tradereso}
\end{figure}\\

To underline their results, the authors use two different permutations of their model to check the model's robustness, which they
confirm by the very similar pattern of the inequality-adjusted gains. Another point of their robustness check is in the form of 
reevaluation of two alternative assumptions about the tariff redistribution. The first assumption of changes is the exclusion of 
tariff revenue from the model, however the correlation of equality gains and income gains is still negative. The second
reevaluation is a change in the compensation of the tariff losses by governmental progressive taxes instead of proportional
taxes, however the pattern does not show any impact and the trade-off countries are still dominated in favor of 
liberalization.
Another form of robustness test is done by using the model to evaluate three different protectionist scenarios instead,
the first one increases the tariffs by 10 percent uniformly, the second is a relative increase of 10 percent and the third
one is an increase to 62.4 percent for all tariffs. However, all three scenarios confirm the results, that in this case
43 to 48 countries inequality-adjusted income gains would decrease with increased protection. This underlines the assumption,
that the majority of countries is actually financially hurt by their ongoing protectionism.
\begin{table}[h!]
    \centering
    \includegraphics*[width=\textwidth]{graphics/protectionist.jpg}
    \caption{Results of three Protectionist Scenarios by Artuc, Rijkers and Porto from 2019}
    \label{table:protectionist}
\end{table}\\
Finally, to summarize their study's results, the authors come to the conclusion, that their model confirmed the negative
correlation between income gains and equality gains, which leads to a trade-off in the case of liberalization. While liberalization
lead to income gains in 45 countries in their study, 9 countries would face losses instead. To finish, they give their thought that
based on their study, the majority of developing countries would profit of liberalization and that the ongoing trend of protectionism
is causing significant welfare losses in developing countries. 