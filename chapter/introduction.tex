\chapter{Introduction}
In today's world with ongoing worldwide economic crises like the COVID-19-pandemic or the current Russia-Ukraine-conflict,
trade policies are seen and used as possible, important instruments to either help or punish specific countries' economies. 
The outcome of those crises in the recent years is that the majority of countries currently suffers a decline in social welfare caused by those ongoing crises. Through new 
developed and improved trade models and the current worldwide economic situation, the interest in the impact of trade policies,
especially the trade liberalization, and the distribution of the gains from trade increased. 
While trade liberalization is commonly seen as economy and welfare enhancing, the majority of developing countries with low social welfare
still tend to restrict their trade in favor of protectionism. To estimate the impact of trade liberalization and the trade-off
between the income gains and inequality costs, Artuc, Rijkers and Porto reviewed recent and past trade models to measure 
income gains and inequality costs for 54 developing countries in their article from 2019. Using household survey data from 54
developing countries in combination with trade data and tariff data, they developed a new model to determine the income gains,
inequality costs and therefore the inequality adjusted income gains of trade. With the model, they were able to answer the
question, whether there are trade-offs between income gains and inequality costs in case of trade liberalization and whether
the trade liberalization is beneficial for the studied countries' welfare.\\

This paper reviews the study by Artuc, Rijkers and Porto from 2019 and takes several critical aspects of the paper itself into consideration.
The first section describes the data used in the study and describes the three parts, the household survey data, the tariff data
and the trade data. The second section summarizes the methods of the study by Artuc et. al. and explains their results.
After that, the next section will showcase several critical aspects in regard of the data used, 
the methods used to form their model and the results, to underline possible weaknesses and strengths of this study. 
Finally, a conclusion is given about the quality of the model and possible improvements for further use cases. 