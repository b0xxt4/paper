\section{Conclusion}

By summarizing and reviewing the study by \cite{Artuc.2019}, the following conclusion can be given regarding
their model, the chosen data set and the results of the study.

While the data set is based on household surveys from 54 developing countries, which range back to 1998 in the most extreme
cases, the data still seems to be the best possible dataset for this use case as especially the chosen developing countries
suffer a lot under the lack of data in other form like administrative data. Therefore, the used dataset meets all the requirements
and can be assumed as the most recent and fitting data for their own developed model. This just showcases, that especially
countries, which would probably benefit by a trade policy change towards freer trade, lack the monitoring and collection of
data or the technology itself, which needs to be improved to increase the quality of possible estimations regarding the 
countries' economy, especially on the scope of households to battle poverty and to push the trade policy into a welfare-enhancing way.\\
However, this can not be said as well for the used COMTRADE data, which is criticized in recent literature for its' aggregation process, 
where a lack of quantity information can lead to higher trade values or even to disconnecting the unit value from the countries context of
production cost and specialization in production. This results in a lower reliability of the used trade units and showcases, why there are
more reliable and accurate databases like the Traded Unit Values database for example. \Cite{Artuc.2019} should have used one of those
databases instead to improve the significance of their results. As the review of the TRAINS database did not show any recent issues or 
alternatives in recent literature, it has to be assumed to be a fitting data for the appliance of the study by \cite{Artuc.2019}. \\

Another specific weakness and also strength of their developed model lies in their use of the extended agricultural household
view as base view, which denies possible analyses regarding standard trade theory like Stolper-Samuelson effects for example.
However, \cite{Artuc.2019} mentioned that they chose this household model to enable the new view of scope of households to get 
results addressing the heterogeneity of households. This problem also appears in the selected data of the household surveys 
itself, as this data lacks information about returns to capital and corporate benefits, which would be needed for several 
models regarding the standard trade theory. The authors also compare their results in robustness tests with other studies, 
where standard trade models were used and showed the same pattern in the results and confirmed the negative correlation between
income gains and equality gains.\\

While the authors of the main study conclude that trade liberalization would be welfare enhancing itself based on the results,
other literature refers to the potential need of complementary policies to achieve the anticipated welfare changes, especially
for the low income households, there might be a need to enable some form of protection and enhance the chance to receive and
exploit the income gains through additional policies. But even though as \cite{Artuc.2019} tend to view trade 
liberalization in isolation in their model, their results underline the potential of trade liberalization and showcases its 
beneficial side. Even critics of the isolated view of trade liberalization underline the statement, that it is one of the
easiest to commit to tools to possibly tackle poverty and enhance social welfare, but with the possibility in the dependence
of complementary policies to enhance the equal and anticipated distribution of the income gains. \\

So in total, the study by \cite{Artuc.2019} showcases a new concept of model to investigate the impact of
trade policies on the income gains and equality gains of countries, which enables the scope of households and their heterogeneity.
Using this model on the best possible data from household surveys, they found out that, liberalization causes
trade-offs between income gains and equality gains. Those trade-offs were explained with the found negative correlation between income gains
and equality gains, which was proven in their study as well. Their second and final conclusion of the results, that trade liberalization is 
welfare enhancing in most cases, seems to be confirmed as well as other literature with standard models tend to show the same conclusion. 
The study gives a concrete direction for most developing countries regarding their trade policy and offers the possibility for the estimation 
of impacts caused by trade policies like the liberalization and the protectionism. 
However, to increase the accuracy and significance of the study, the temporal restriction of the household surveys needs to be updated
and better data for the trade and value data has to be used to come to more recent and significant conclusions regarding the question, 
whether or whether not developing countries should move towards trade liberalization instead of the ongoing protectionism in the majority
of countries.





